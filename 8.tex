\documentclass{article}
\usepackage{ctex}
\usepackage[colorlinks=true,linkcolor=blue]{hyperref}
\usepackage[a4paper,scale=0.8]{geometry}
\usepackage{amsmath,mathtools}
\usepackage{amssymb}
\usepackage{lettrine}
\usepackage{listings}
\usepackage{graphicx}
\usepackage{makeidx}   \makeindex
\usepackage{caption} 
\usepackage{tikz}
\usepackage{amsthm}
\newcommand{\dif}{\mathop{}\!\mathrm{d}}
\newcommand{\R}{\mathbb{R}}
\newcommand{\C}{\mathbb{C}}
\newcommand{\Z}{\mathbb{Z}}
\newcommand{\e}{\mathrm{e}}
\newcounter{myenumi}
\newenvironment{en}{
    \begin{list}{\arabic{myenumi}).}{\usecounter{myenumi}
        \setlength{\itemsep}{1ex}
        \setlength{\topsep}{1ex}}
}{\end{list}}



\title{数学分析作业}
\date{\today}
\begin{document}
\maketitle
\tableofcontents
\newpage
\section{第八章作业}
\subsection{简答题}
\subsubsection{1}
\begin{enumerate}
\item 求下列不定积分\footnote{以上所以题目均可在mathematica或者\url{https://www.wolframalpha.com/}找到答案}:
\begin{en}
\item $\int (x-\frac{1}{\sqrt{x}})^2\dif x$
\item $\int (2^x+3^x)^2\dif x$
\item $\int\frac{3}{\sqrt{4-4x^2}}\dif x$
\end{en}
\begin{proof}
\begin{flalign*}
    &\int (x-\frac{1}{\sqrt{x}})^2\dif x&&\int (2^x+3^x)^2\dif x &&\int\frac{3}{\sqrt{4-4x^2}}\dif x\\
    ={}&\int (x^2-2\sqrt{x}+\frac{1}{x})\dif x   &={}&\int(4^x+9^x+2\cdot6^x)\dif x   &={}&\int(\frac32\frac{1}{\sqrt{1-x^2}})\dif x  \\
    ={}&\frac{x^3}{3}-\frac{4}{3}x^{\frac32}+\ln(|x|)+C &={}&\frac{4^x}{\ln(4)}+\frac{9^x}{\ln(9)}+\frac{2\cdot6^x}{\ln(6)}+C&={}&\frac32\arcsin(x)+C
\end{flalign*}
\end{proof}
\item \begin{en}
    \item $\int \frac{x^2}{3(1+x^2)}\dif x$
    \item  $\int \tan^2(x)\dif x$
    \item $\int \sin^2(x)\dif x$
    \item $\int \frac{\cos(2x)}{\cos(x)-\sin(x)}\dif x$
\end{en}
\begin{proof}
\begin{flalign*}
&\int \frac{x^2}{3(1+x^2)}\dif x&&\int \tan^2(x)\dif x&&\int \frac{\cos(2x)}{\cos(x)-\sin(x)}\dif x\\
={}&\int\frac13(1-\frac{1}{1+x^2})\dif x&={}&\int \sec^2(x)-1\dif x&={}&\int\cos(x)+\sin(x)\dif x\\
={}&\frac13{x-\arctan(x)}+C&={}&\tan(x)-x+C&={}&\sin(x)-\cos(x)+C
\end{flalign*}
\end{proof}
\item \begin{proof}\begin{flalign*}
&\int \frac{\cos(2x)}{\cos^2(x)\sin^2(x)}\dif x&&\int \sqrt{x\sqrt{x}\sqrt{x}}\dif x&&\int (\sqrt{\frac{1+x}{1-x}+\frac{1-x}{1+c}})\dif x\\
={}&2 x-\frac{1}{2} \sin (2 x)-\tan (x)+C&={}&\frac{8 x^{15/8}}{15}+C&={}&2\arcsin(x)+C\\
&\int \cos(2x)\cos(x)\dif x &&\int \frac{\sqrt{x^4+x^{-4}+2}}{x^3}\dif x&&\\
={}&\frac12(\sin(x)+\frac{\sin(3x)}{3})+C&={}&\ln(x)+\frac{1}{-4x^4}+C&&
\end{flalign*}
\end{proof}
\item
\begin{proof}\begin{align*}
&\int e^{-|x|}\dif x\\
={}&-sgn(x)\cdot e^{|x|}+C
\end{align*}
\end{proof}
\item $f'(\arctan(x))=x^2$,求$f(x):$
\begin{proof}
    $f'=\tan^2(x)\to f'=\sec^2(x)-1\to f=\tan(x)-x+C$
\end{proof}
\item \begin{proof}
\begin{flalign*}
    &\int \frac{\dif x}{1+2x}&&\int (\frac{1}{3-x^2}+\frac{1}{1-3x^2})\dif x&&\int 2^(2x+3)\dif x\\
    ={}&\frac{\ln(2x+1)}{2}+C&={}&\arcsin(\frac{x}{\sqrt{3}})+\frac{\arcsin(\sqrt{3}x)}{\sqrt{3}}+C&={}&\frac{2^{2x+2}}{\ln(2)}+C\\
    &\int \frac{\dif x}{\sqrt[3]{7-5x}}&&&&\\
    ={}&-\frac{3\cdot(7-5x)^{\frac23}}{10}+C&&&&
\end{flalign*}
\end{proof}
\item \begin{proof}
\begin{flalign*}
&\int \frac{\dif x}{\sin^2(2x+\frac{\pi}{4})}&&\int\frac{\dif x}{1+\cos x}&&\int \csc x\dif x\\
={}&-\frac{\cot(2x+\frac{\pi}{4})}{2}+C&={}&\tan(\frac{x}{2})+C&={}&-\tanh ^{-1}(\cos (x))+C\\
&\int \frac{x}{\sqrt{1-x^2}}\dif x&&&&\\
={}&-\sqrt{1-x^2}+C&&&&
\end{flalign*}
\end{proof}
\end{enumerate}
\section{第9章作业}
\subsection{1}
\begin{align*}
\int_0^1\e^{x}\dif x&=\sum_{k=0}^n\e^{\frac{k}{n}}\cdot\frac{1}{n}\\
&=\frac{1-\e}{1-\e^{\frac{1}{n}}}\cdot\frac1n\\
&=\e-1
\end{align*}
\subsection{2}
\begin{enumerate}
    \item \begin{align*}
\int_0^1\frac{1-x^2}{1+x^2}\dif x&=\int_0^1\frac{2}{1+x^2}-1\dif x\\
&=2\arctan(1)-1\\
&=\frac{\pi}{2}-1
\end{align*}
\item
\begin{align*}
\int_0^1\frac{\e^x-\e^{-x}}{2}\dif x&=\left.、frac{\e^{x}+\e^{-x}}{2}\right|_0^1\\
&=\frac{\e+\frac{1}{\e}-2}{2}
\end{align*}
\item
\begin{align*}
    \int_4^9(\sqrt{x}+\frac{1}{\sqrt{x}})\dif x&=\left.\frac{3}{2}x^{\frac32}-2\sqrt{x}\right|_4^9\\
&=\frac{77}{2}
\end{align*}
\item 
\begin{align*}\int_{\frac1\e}^\e\frac1x(\ln(x))^2\dif x&=\left.\frac{\ln(x)^3}{3}\right|_{\frac1\e}^\e\\
&=\frac{8}{3}
    \end{align*}
\end{enumerate}
\subsection{3}
\begin{enumerate}
    \item \begin{align*}\lim_{n\to\infty}\frac{1+2^3+\cdots+n^3}{n^4}&=\int_0^1x^3\dif x\\
&=\left.\frac{x^4}{4}\right|_0^1\\
&=\frac14
\end{align*}
\item \begin{align*}
    \lim_{n\to\infty}n(\frac1{n^2+1}+\frac1{n^2+2^2}+\cdots+\frac1{n^2+n^2})&=\int_0^1\frac{1}{1+x^2}\dif x\\
&=\left.\arctan(x)\right|_0^1\\
&=\frac{\pi}{4}
\end{align*}
\end{enumerate}
\subsection{4}
\begin{proof}
证明:若$T'$是$T$增加若干分点后所得的分割,则$\sum_{T'}\omega'_i\Delta x_i'\le\sum_T\omega'_i\Delta x_i$.\\
我们不妨设$T'$增加的点在$x_i,x_{i+1}$之间,如果取到边界相当于没增加分点,不妨设这个点为$y_i$,先假设只增加这一个点,后面我们会证明,增加一个点,那么这个doubour和会下降,因此我们可以先假定为一个点,
那么与原先不同的doubour和只有$x_i,y_i,x_{i+1}$这三点,我们只需要证明:\[
w'_i(y_i-x_i)+w'_{i+1}(x_{i+1}-y_i)\le \omega_i(x_{i+1}-x_i)\]
然后我们又有$w'_i,w'_{i+1}\le \omega_i$,以及$(y_i-x_i)+(x_{i+1}-y_i)=(x_{i+1}-x_i)$,所以这就证明完了。
因此每增加一个分点这个doubour和会缩小,因此增加若干分点后,会比增加一个分点还小。这就是我们的证明。
\end{proof}
\subsection{5}
\begin{proof}
任取$\varepsilon>0$,由于$f$在$[a,b]$可积,存在一个分割$T$,使得$\sum_T\omega_i\Delta x_i<\varepsilon$,如果这个分割点形成的区间包含了$\alpha$或者$\beta$,我们根据上一个命题,我们可以选取$\alpha$以及$\beta$为分点。就有分割$T'$:\[
\sum_{[\alpha,\beta]}\omega_i\Delta x_i\le\sum_{T'}\omega_i\Delta x_i\le\sum_{T}\omega_i\Delta x_i\le \varepsilon\]
\end{proof}
\subsection{6}
\begin{proof}
我们不妨设$h=f-g$,根据题意$h$在$[a,b]$只有有限个点(不妨设为$N$)不为0,但是有界(不妨设为$M$),对任意$\varepsilon>0$,取等距分割$T=\{x_0,x_1,\cdots,x_n\}$,使得$\Delta x_i=\frac{b-a}{n}$,那么我们有\[
\sum_T\omega_i\Delta x_i\le\frac{b-a}{n}NM\]
对于$n\to\infty$时右侧是为$0$的,所以对于$\varepsilon>0$,存在一个$N'$当$n>N'$就小于$\varepsilon$.
\end{proof}
\subsection{7}
\begin{enumerate}
\item $\int_0^{\frac\pi2}x\dif x=\frac12(\frac{\pi^2}{4})=\frac{\pi^2}{8}$
\item $\int_0^{\frac\pi2}\sin(x)\dif x=1$
\item 我们又有在$[0,\frac\pi2]$上,$x\ge\sin(x)$,所以我们有$\frac{\pi^2}{8}>1$
\end{enumerate}
\subsection{8}
\begin{enumerate}
    \item $\e^{x^2}$在$[0,1]$是单调的,取两边就得到答案
    \item $g(x)=\frac{\ln(x)}{\sqrt{x}}$,我们有$g'(x)=\frac{1-\frac12\ln(x)}{x^{\frac32}}$。所以$g\le g(e^2)=\frac2\e$,右侧就完成了。左侧的话,求出$g$在这个区间的最小值(端点$\e$)$\frac1{\sqrt{\e}}$,然后完成。
\end{enumerate}
\subsection{9}
我们先给出一个结论:\[
\frac{\dif\int_0^{h(x)}g(x,t)\dif t}{\dif x}=\int_0^{h(x)}g_x'(x,t)\dif t+g(x,h(x))h'(x)\]
\begin{align*}
F'(x)&=\int_0^x f(t)\dif t\\
F''(x)&=f(x)\\
\end{align*}
所以完成了。
\subsection{10}
\begin{enumerate}
    \item 求导,$1$
    \item 求导,$0$
\end{enumerate}
\subsection{11}
\begin{enumerate}
\item 把$\sin(2x)$拆出来,然后把$\sin(x)$放到$\dif$里面,答案是$\frac27$
\item 换元$x=2\sin(t)$,$\frac\pi3+\frac{\sqrt{3}}2$
\item $\frac{2}{3} \left(\sqrt{3}-1\right)$
\item $\arctan(\e)-\frac\pi4$
\item $\frac{1}{2} (\pi -2)$
\item $\frac{1}{2} \left(1+\e^{\pi /2}\right)$
\item $\e$
\item $2$
\item $\frac\pi4$
\end{enumerate}



























\end{document}